\documentclass{beamer}

\newcommand{\hide}{\onslide+<+(1)->}

\usepackage[utf8]{inputenc}
\usepackage[TS1,T1]{fontenc}
\usepackage[ngerman]{babel}
\usepackage{listings}
\usepackage{xcolor}
\usepackage{tikz}
\usetikzlibrary{backgrounds,decorations.pathreplacing}
\usepackage{mathpartir}
\usepackage{adjustbox}
\usepackage{amsmath}
\usepackage{mathtools}
\usepackage[safe]{tipa} % for \textlambda
\usepackage{wasysym}
\usepackage{stmaryrd}

\newcommand\pfun{\mathrel{\ooalign{\hfil$\mapstochar\mkern5mu$\hfil\cr$\to$\cr}}}


\usetheme[titlepage0]{KIT}


\definecolor{light-gray}{gray}{0.95}
\lstdefinestyle{haskell}{
        ,columns=flexible
        ,basewidth={.365em}
        ,keepspaces=True
	,belowskip=0pt
	,backgroundcolor=\color{light-gray}
	,frame=single
	,xleftmargin=2em
	,xrightmargin=2em
        ,basicstyle=\small\sffamily
        ,stringstyle=\itshape
	,language=Haskell
        ,texcl=true
        ,showstringspaces=false
        ,keywords={module,where,import,data,let,in,case,of}
}
\lstnewenvironment{haskell}{\lstset{style=haskell}}{}


\title{Launchbury's Semantik für Lazy Evaluation}
%\KITtitleimage[viewport=0 400 2000 100]{probeklausur}
\author{Joachim Breitner}
\date{27.11.2012}
\iflanguage{ngerman}{%
  \institute{LEHRSTUHL PROGRAMMIERPARADIGMEN}%
}{%
  \institute{PROGRAMMING PARADIGMS GROUP}%
}


\begin{document}

\maketitle


\begin{frame}{Übersicht}%
Launchbury (POPL 1993, 168 Zitationen\footnote{laut CiteSeer})
\begin{itemize}
\item entwickelt eine operationale Semantik für funktionale Sprachen, die Bedarfsauswertung und Sharing darstellt,
\item gibt dazu eine denotationale Semantik an,
\item und beweist dass erstere korrekt bezüglich zweiterer ist.
\end{itemize}

\bigskip
\pause


Joachim (2012)
\begin{itemize}
\item will dies in Isabelle formalisieren,
\item findet einen Fehler im Paper,
\item bastelt eine andere, aber äquivalente operationale Semantik,
\item und beweist die Korrektheit.
\end{itemize}
\end{frame}

\newcommand{\sVar}{\text{Var}}
\newcommand{\sExp}{\text{Exp}}
\newcommand{\sHeap}{\text{Heap}}
\newcommand{\sVal}{\text{Val}}
\newcommand{\sValue}{\text{Value}}
\newcommand{\sEnv}{\text{Env}}
\newcommand{\sApp}[2]{\operatorname{#1}#2}
\newcommand{\sLam}[2]{\text{\textlambda} #1.\, #2}
\newcommand{\sLet}[2]{\text{\textsf{let}}\ #1\ \text{\textsf{in}}\ #2}
\newcommand{\sred}[4]{#1 : #2 \Downarrow #3 : #4}
\newcommand{\sRule}[1]{\text{{\textsc{#1}}}}
\newcommand{\fv}[1]{\text{fv}(#1)}
\newcommand{\dom}[1]{\text{dom}\,#1}
\newcommand{\fresh}[1]{#1'}

\begin{frame}
\frametitle{Terme}

\begin{alignat*}{2}
x,y,u,v,w &\in \sVar
\displaybreak[1]
\\
e &\in
\sExp &&\Coloneqq
\begin{aligned}[t]&
\sLam x e
\mid \sApp e x
\mid x \mid
\\&
\sLet {x_1 = e_1,\ldots,x_n = e_n} e
\end{aligned}
\displaybreak[1]\\
\Gamma, \Delta, \Theta &\in \sHeap &&= \sVar \pfun \sExp
\displaybreak[1]\\
z &\in \sVal &&\Coloneqq \sLam x e
\end{alignat*}
\end{frame}

\begin{frame}
\frametitle{Launchbury’s operationale Semantik}

\begin{mathpar}
\inferrule
{ }
{\sred{\Gamma}{\sLam xe}{\Gamma}{\sLam xe}}
\sRule{Lam}
\\
\inferrule
{\sred{\Gamma}e{\Delta}{\sLam y e'}\\ \sred{\Delta}{e'[x/y]}{\Theta}{z}}
{\sred\Gamma{\sApp e x}\Theta z}
\sRule{App}
\and
\inferrule
{\sred\Gamma e \Delta z}
{\sred{\Gamma, x\mapsto e} x {\Delta, x\mapsto z}{z}}
\sRule{Var}
\and
\inferrule
{\sred{\Gamma,x_1\mapsto e_1,\ldots,x_n\mapsto e_n} e \Delta z}
{\sred{\Gamma}{\sLet{x_1 = e_1,\ldots, x_n = e_n}e} \Delta z}
\sRule{Let}
\end{mathpar}
\end{frame}


\newcommand{\dsem}[2]{\llbracket #1 \rrbracket_{#2}}
\newcommand{\esem}[1]{\{\!\!\{#1\}\!\!\}}

\begin{frame}
\frametitle{Launchbury’s denotationale Semantik}

Semantische Domäne:
\begin{align*}
\sValue &= (\sValue \to \sValue)_\bot &
\sEnv &= \sVar \pfun \sValue
\end{align*}

Bedeutung von Ausdrücken:
\begin{align*}
\dsem{\cdot}{\cdot} &\colon \sExp \to \sEnv \to \sValue \\
\dsem{\sLam xe}{\rho} &=
	%\operatorname{Fn}
	(\lambda v. \dsem{e}{\rho \sqcup \{x \mapsto v\}})\\
\dsem{\sApp e x}\rho &= \dsem e\rho\
	%\downarrow_{\operatorname{Fn}}
	\dsem x \rho \\
\dsem{\sVar x}\rho &= \rho (x)\\
\dsem{\sLet{x_1 = e_1,\ldots, x_n = e_n}e}\rho &= \dsem{e}{\esem{ x_1\mapsto e_1,\ldots,x_n\mapsto e_n}\rho}
\end{align*}

Bedeutung von Heaps:
\begin{align*}
\esem{\cdot}\cdot &\colon \sHeap \to \sEnv \to \sEnv \\
\esem{ x_1\mapsto e_1,\ldots,x_n\mapsto e_n}\rho
&= \mu \rho'. \rho \sqcup (x_1 \mapsto \dsem{e_1}{\rho'}) \sqcup \cdots \sqcup (x_n \mapsto \dsem{e_n}{\rho'})\\
\esem{\Gamma} &= \esem{\Gamma}{\rho_\bot}
\end{align*}
\end{frame}

\begin{frame}
\frametitle{Launchbury’s Theorem 2}

Korrektheit der operationalen Semantik:
\[
\sred \Gamma  e \Delta z \implies \dsem{e}{\esem{\Gamma}} = \dsem{z}{\esem{\Delta}}
\]

Verallgemeinerte Aussage (theorem 2):
\[
\forall \rho.\ \sred \Gamma  e \Delta z \implies \dsem{e}{\esem{\Gamma}\rho} = \dsem{z}{\esem{\Delta}\rho} \wedge {\esem{\Gamma}\rho} \le \esem{\Delta}\rho
\]

\pause
Leider falsch. Seien
\begin{align*}
e &= x & z &= \sLam \_ {\sLet {y = y}y} &  \Gamma = \Delta &= x \mapsto z  & \rho &= x \mapsto \lambda \_.\operatorname{id},
\end{align*}
dann ist
\begin{align*}
\dsem{e}{\esem{\Gamma}\rho} &= \esem{\Gamma}\rho\ (x) = \rho\ (X) \sqcup \dsem{z}{\esem{\Gamma}\rho} = (\lambda \_.\operatorname{id}) \sqcup (\lambda \_.\bot ) =  (\lambda \_.\operatorname{id})\\
\intertext{aber}
\dsem{z}{\esem{\Delta}\rho} &= (\lambda \_.\bot ) \ne (\lambda \_.\operatorname{id}).
\end{align*}
\end{frame}

\newcommand{\newstuff}[1]{%
	\onslide<1,3>{{%
	\textcolor<3>{blue}{%
		#1{}%
	}}}%
	}
\begin{frame}
\frametitle{Meine Semantik}

\begin{mathpar}
\inferrule
{ }
{\sred
{\Gamma}{\newstuff{u \mapsto }\sLam xe \newstuff{, \Gamma'}}
{\Gamma}{\newstuff{u \mapsto {}}\sLam xe \newstuff{, \Gamma'}}
}
\sRule{Lam}
\\
\inferrule
{\sred{\Gamma}{\newstuff{v\mapsto }e\newstuff{,u \mapsto \sApp v x, \Gamma'}}{\Delta}{\newstuff{v \mapsto }\sLam y e'\newstuff{, u \mapsto \sApp v x, \Delta'}}\\
\sred{\Delta}{\newstuff{u \mapsto }e'[x/y]\newstuff{, \Delta'}}{\Theta}{\newstuff{u \mapsto }z\newstuff{, \Theta'}}}
{\sred\Gamma{\newstuff{u \mapsto {}}\sApp e x\newstuff{, \Gamma'}}{\Theta}{\newstuff{u \mapsto }z\newstuff{, \Theta'}}}
\sRule{App}
\and
\inferrule
{\sred\Gamma {\newstuff{x \mapsto }e\newstuff{, u \mapsto x, \Gamma'}}{\Delta}{\newstuff{x \mapsto }z\newstuff{, u \mapsto z, \Delta'}}}
{\sred{\Gamma, x\mapsto e} {\newstuff{u \mapsto }x\newstuff{, \Gamma'}} {\Delta, x\mapsto z}{\newstuff{u \mapsto }z\newstuff{, \Delta'}}}
\sRule{Var}
\and
\inferrule
{\sred{\Gamma,x_1\mapsto e_1,\ldots,x_n\mapsto e_n} {\newstuff{u \mapsto }e\newstuff{, \Gamma'}} \Delta {\newstuff{u \mapsto }z\newstuff{, \Delta'}}}
{\sred{\Gamma}{\newstuff{u \mapsto }\sLet{x_1 = e_1,\ldots, x_n = e_n}e\newstuff{, \Gamma'}} \Delta {\newstuff{u \mapsto }z\newstuff{, \Delta'}}}
\sRule{Let}
\end{mathpar}
\end{frame}

\begin{frame}
\frametitle{Meine Korrektheitsaussage}

Laut Isabelle gilt:
\[
\sred{\Gamma}{\Gamma'}{\Delta}{\Delta'} \implies \esem{\Gamma, \Gamma'} \le \esem{\Delta, \Delta'}
\]

und damit auch

\[
\sred \Gamma  e \Delta z \implies \dsem{e}{\esem{\Gamma}} = \dsem{z}{\esem{\Delta}} \wedge {\esem{\Gamma}} \le \esem{\Delta}.
\]

\end{frame}


\begin{frame}
\frametitle{Und jetzt\dots}

\begin{itemize}
\item Isabelle-Theorien aufräumen\\{\small(zur Zeit 9k nicht-leere Zeilen, davon viel unnötig)}
\item AFP-Submission
\item Adäquatheit zeigen?
\item Journal-Paper?
\end{itemize}
\end{frame}







\end{document}
